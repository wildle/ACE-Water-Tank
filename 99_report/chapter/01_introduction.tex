\chapter{Introduction}

\begin{figure}[h!]
    \centering
    \includegraphics[width=0.35\textwidth]{img/WaterTank.png}
    \caption{Water Tank laboratory system}
    \label{fig:wt_system}
\end{figure}

This laboratory experiment addresses the modeling and control of a nonlinear hydraulic two-tank system. 
The setup consists of a pump that transfers water from a basin into an upper reservoir, from which the 
water flows into a lower reservoir before returning to the basin. The system dynamics are governed by 
mass conservation and nonlinear flow relations caused by gravity-driven outflows.

The objective of the laboratory is to regulate the water level of the lower reservoir by controlling 
the pump input. To achieve this, a nonlinear state-space model of the system is derived and linearized 
around a suitable operating point. Based on the linearized model, a PID controller is designed and 
validated in simulation. The nonlinear plant model is used to assess closed-loop performance under 
practical constraints such as actuator saturation, measurement noise, and physical limitations of 
the tank levels. In addition, a flatness-based feed-forward control approach is investigated to improve 
reference tracking during trajectory transitions.

MATLAB/Simulink is used as the primary environment for modeling, controller design, and simulation, 
serving as preparation for the subsequent implementation on real laboratory hardware.

The report is structured as follows:
\begin{itemize}
    \item Chapter~2 summarizes the theoretical background, including system modeling and linearization.
    \item Chapter~3 describes the laboratory setup and the parameter identification procedures.
    \item Chapter~4 presents the simulation models and controller design.
    \item Chapter~5 discusses the obtained results and their interpretation.
    \item Chapter~6 concludes the report and highlights key findings.
\end{itemize}

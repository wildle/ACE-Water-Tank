\chapter{Laboratory Setup and Parameter Identification}

The laboratory setup is used for experimental parameter identification of the hydraulic
two-tank system and comprises a pump supplying the upper reservoir as well as pressure sensors
for water level measurement.

\paragraph{Pump identification}
The pump generates the inlet mass flow to the upper reservoir and is driven by an input
voltage. Based on the characteristic curve provided in the laboratory script, the pump is
approximated as a static linear system
\begin{equation}
\dot{m} = K_{\mathrm{pump}} (U_{\mathrm{in}} - U_0),
\end{equation}
where the gain $K_{\mathrm{pump}}$ and the voltage offset $U_0$ are determined by linear
regression of the measured data.

\paragraph{Pressure sensor calibration}
The water level in each reservoir is measured by a pressure sensor. Since hydrostatic pressure
is proportional to the water column height, a linear sensor model is assumed. Calibration is
performed by recording the sensor output voltage at known water heights and identifying the
corresponding gain and offset parameters.

\paragraph{Hydraulic parameters and constraints}
The hydraulic outflow behaviour of the reservoirs is characterized by an effective flow
coefficient, which is identified experimentally using steady-state measurements. System
operation is constrained by the maximum admissible pump voltage and by the physical limits of
the reservoir heights. These constraints are accounted for in the simulation model.


Test
\chapter{Laboratory Setup and Parameter Identification}
\label{chapter:LaboratorySetupAndParameterIdentification}
\enlargethispage{3.5\baselineskip}

\section{Parameter identification}
\subsection{Pump parameter identification}
The pump generates the inlet mass flow to the upper reservoir and is driven by an input
voltage. Based on the characteristic curve provided in the laboratory script \cite{ACE_WT_Lab}, the pump is
approximated as a static linear system
\begin{equation}
	K_{\mathrm{pump}} = \frac{\dot{m}}{(U_{\mathrm{in}} - U_0)} 
\end{equation}
where the gain $K_{\mathrm{pump}}$ and the voltage offset $U_0$ are determined by linear
regression of the diagram, given in the laboratory script \cite{ACE_WT_Lab}, depicted in \ref{fig:pumpcurve}.
\begin{figure}[H]
	\centering
	\includegraphics[height=0.4\textheight]{img/pump_curve.pdf}
	\caption[Transition in the Simulation]{Transition from \qty{0.1}{\meter} to \qty{0.4}{\meter} using the Simulink-Simulation and the non-linear model of the system}
	\label{fig:pumpcurve}
\end{figure}
The linear regression of the pump diagram in \ref{fig:pumpcurve} returns following values for the corresponding parameters: $K_{\mathrm{pump}} = \qty{0.0167}{\kilogram\per\second\per\volt}$ and $U_0 = \qty{0.25}{\volt}$
\subsection{Water tank parameter identification}
\section{Sensor calibration}
\begin{figure}[H]
	\centering
	\includegraphics[height=.4\textheight]{img/sensor2_calibration.pdf}
	\caption[Transition in the Simulation]{Transition from \qty{0.1}{\meter} to \qty{0.4}{\meter} using the Simulink-Simulation and the non-linear model of the system}
	\label{fig:sensorcal}
\end{figure}






The laboratory setup is used for experimental parameter identification of the hydraulic
two-tank system and comprises a pump supplying the upper reservoir as well as pressure sensors
for water level measurement.


where the gain $K_{\mathrm{pump}}$ and the voltage offset $U_0$ are determined by linear
regression of the measured data.

\paragraph{Pressure sensor calibration}
The water level in each reservoir is measured by a pressure sensor. Since hydrostatic pressure
is proportional to the water column height, a linear sensor model is assumed. Calibration is
performed by recording the sensor output voltage at known water heights and identifying the
corresponding gain and offset parameters using linear regression.

\paragraph{Hydraulic parameters and constraints}
The hydraulic outflow behaviour of the reservoirs is characterized by an effective flow
coefficient, which is identified experimentally using steady-state measurements. System
operation is constrained by the maximum admissible pump voltage and by the physical limits of
the reservoir heights.

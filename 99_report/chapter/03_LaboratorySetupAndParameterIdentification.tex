\chapter{Laboratory Setup and Parameter Identification}
\label{chapter:LaboratorySetupAndParameterIdentification}
\enlargethispage{3.5\baselineskip}
\section{Parameter identification}
\subsection{Pump parameter identification}
The pump generates the inlet mass flow to the upper reservoir and is driven by an input
voltage. Based on the characteristic curve provided in the laboratory script \cite{ACE_WT_Lab}, the pump is
approximated as a static linear system:
\begin{equation}
	K_{\mathrm{pump}} = \frac{\dot{m}}{(U_{\mathrm{in}} - U_0)} 
\end{equation}
where the gain $K_{\mathrm{pump}}$ and the voltage offset $U_0$ are determined by linear
regression of the diagram, given in the laboratory script \cite{ACE_WT_Lab}, depicted in figure \ref{fig:pumpcurve}.
\begin{figure}[H]
	\centering 
	\includegraphics[height=0.4\textheight]{img/pump_curve.pdf}
	\caption[Diagram of the linear pump relationship]{Diagram of the linear pump relationship given the laboratory script  \cite{ACE_WT_Lab}}
	\label{fig:pumpcurve}
\end{figure}
The linear regression of the pump diagram in \ref{fig:pumpcurve} returns following values for the corresponding parameters: $K_{\mathrm{pump}} = \qty{0.0167}{\kilogram\per\second\per\volt}$ and $U_0 = \qty{0.25}{\volt}$
\subsection{Water tank parameter identification}
The hydraulic outflow behaviour of the reservoirs is characterized by an effective flow
coefficient, which is identified experimentally using steady-state measurements. System
operation is constrained by the maximum admissible pump voltage and by the physical limits of
the reservoir heights. $a_{1}$ can be calculated by:
\begin{equation}
	a_{1} = \frac{b \cdot k_{\text{pump}} (v_{\text{input}} - v_0)}{\sqrt{h_{1}}}
\end{equation}
where $b$ is:
\begin{equation}
	b = \frac{1}{\rho A_t} = \frac{1}{\qty{1000}{kg/m^3} \times \qty{50e-4}{m^2}} = \qty{0.2}{m/kg}
\end{equation}
and $v_{\text{input}}$ and $h_{1}$ are vectors, therefore the mean has to be taken. Resulting in:
\begin{equation}
	a_{1, \text{final}} = \text{mean}(a_{1}) = \qty{0.0574}{m^{1/2}/s}
\end{equation}
$a_{2}$ can be calculated by:
\begin{equation}
	a_{2} = a_{1} \cdot \frac{\sqrt{h_{1}}}{\sqrt{h_{2}}}
\end{equation}
where $h_{1}$ and $h_{2}$ are vectors, therefore the mean has to be taken. Resulting in:
\begin{equation}
	a_{2, \text{final}} = \text{mean}(a_{2}) = \qty{0.0524}{m^{1/2}/s}
\end{equation}

\section{Sensor calibration}
The water level in each reservoir is measured by a pressure sensor. Since hydrostatic pressure
is proportional to the water column height, a linear sensor model is assumed. Calibration is
performed by recording the sensor output voltage at known water heights and identifying the
corresponding gain and offset parameters using linear regression.
\begin{figure}[H]
	\centering
	\includegraphics[height=0.4\textheight]{img/sensor2_calibration.pdf}
	\caption[Transition in the Simulation]{Diagram of the linear fit of the measurements to determine} using the Simulink-Simulation and the non-linear model of the system}
	\label{fig:sensorcal}
\end{figure}
The linear regression of calibration of the sensor of the second water tank, which is depicted in diagram \ref{fig:sensorcal} returns following values for the corresponding parameters: $K_{\text{sens}} = \qty{0.1394}{\meter\per\volt}$ and $U_{0, \text{sens}} = \qty{2.0866}{\volt}$


\chapter{Laboratory Setup and Parameter Identification}
\label{chapter:LaboratorySetupAndParameterIdentification}
\enlargethispage{3.5\baselineskip}

The laboratory setup consists of a two-tank hydraulic system equipped with a pump and pressure
sensors for water level measurement. Water is pumped from a basin into the upper reservoir,
while gravity-driven flow connects the upper and lower reservoirs. The water level of the lower
reservoir is used as the controlled output.

Control and data acquisition are handled via National Instruments hardware. The system model
and control structure are implemented in MATLAB and Simulink and deployed to the laboratory
hardware using MATLAB Coder. Physical constraints such as pump voltage saturation and the
maximum admissible reservoir heights impose strict bounds on system operation and must be
considered during parameter identification and control design.

\section{Parameter identification}

\subsection{Pump parameters}
The pump provides the inlet mass flow to the upper reservoir and is driven by an input voltage.
Based on the characteristic curve given in \cite{ACE_WT_Lab}, the pump is approximated by a
static linear model

\begin{equation}
	K_{\mathrm{pump}} = \frac{\dot{m}}{(U_{\mathrm{in}} - U_0)} 
\end{equation}

where the gain $K_{\mathrm{pump}}$ and the voltage offset $U_0$ are determined by linear
regression of the diagram, given in the laboratory script \cite{ACE_WT_Lab}, depicted in figure \ref{fig:pumpcurve}.

\begin{figure}[H]
	\centering 
	\includegraphics[height=0.4\textheight]{img/pump_curve.pdf}
	%\caption[Diagram of the linear pump relationship]{Diagram of the linear pump relationship given the laboratory script  \cite{ACE_WT_Lab}}
	\caption{Pump characteristic curve and linear fit \cite{ACE_WT_Lab}}
	\label{fig:pumpcurve}  
\end{figure}

%The linear regression of the pump diagram in \ref{fig:pumpcurve} returns following values for the 
%corresponding parameters: $K_{\mathrm{pump}} = \qty{0.0167}{\kilogram\per\second\per\volt}$ 
%and $U_0 = \qty{0.25}{\volt}$

Linear regression yields $K_{\mathrm{pump}} = \qty{0.0167}{\kilogram\per\second\per\volt}$ and
$U_0 = \qty{0.25}{\volt}$.

\subsection{Water tank parameters}
The hydraulic outflow behaviour of the reservoirs is described by effective flow coefficients
derived from the nonlinear model. These parameters are identified under steady-state operating
conditions, where the water levels remain constant and the inflow equals the outflow. System
operation is constrained by the admissible pump voltage and the physical height limits of the
reservoirs.

Under steady-state conditions, the outflow coefficient of the upper reservoir is obtained as
\begin{equation}
	a_{1} = \frac{b \, K_{\mathrm{pump}} (U_{\mathrm{in}} - U_0)}{\sqrt{h_{1}}},
\end{equation}
with
\begin{equation}
	b = \frac{1}{\rho A_t}
	= \frac{1}{\qty{1000}{kg/m^3} \times \qty{50e-4}{m^2}}
	= \qty{0.2}{\meter\per\kilogram}.
\end{equation}
Evaluating this expression using multiple steady-state measurements yields
\begin{equation}
	a_{1} = \qty{0.0574}{\meter^{1/2}\per\second}.
\end{equation}

The outflow coefficient of the lower reservoir follows from the steady-state relation between
both tank levels,
\begin{equation}
	a_{2} = a_{1} \frac{\sqrt{h_{1}}}{\sqrt{h_{2}}},
\end{equation}
resulting in
\begin{equation}
	a_{2} = \qty{0.0524}{\meter^{1/2}\per\second}.
\end{equation}


\subsection{Sensor parameters}
Water levels in the reservoirs are measured using pressure sensors. Owing to the proportional
relation between hydrostatic pressure and water column height, the sensor behaviour is modeled
as a linear static mapping between output voltage and water level.

Sensor gain and voltage offset are identified from steady-state measurements by linear
regression of the calibration data shown in Figure~\ref{fig:sensorcal}.

\begin{figure}[H]
	\centering
	\includegraphics[height=0.4\textheight]{img/sensor2_calibration.pdf}
	\caption{Pressure sensor characteristic and linear fit}
	\label{fig:sensorcal}
\end{figure}

For the pressure sensor of the second reservoir, the identified parameters are
$K_{\text{sens}} = \qty{0.1394}{\meter\per\volt}$ and
$U_{0,\text{sens}} = \qty{2.0866}{\volt}$.

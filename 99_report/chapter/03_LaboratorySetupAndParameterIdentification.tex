\chapter{Laboratory Setup and Parameter Identification}

The laboratory setup consists of a hydraulic two-tank system including a water basin, a pump,
two vertically arranged reservoirs, and pressure sensors for water level measurement. Water
is pumped from the basin into the upper reservoir and flows sequentially into the lower
reservoir before returning to the basin. The control input is the pump voltage, while the
controlled output is the water level of the lower reservoir.

The experimental system is interfaced via National Instruments hardware and operated using
MATLAB/Simulink in external mode. This setup allows real-time data acquisition, controller
implementation, and parameter identification under realistic operating conditions. Physical
constraints such as maximum tank height and limited pump voltage must be respected throughout
the experiment.

\paragraph{Pump model identification}
The pump generates the inlet mass flow to the upper reservoir and is driven by an input
voltage. Based on the characteristic curve provided in the laboratory documentation, the pump
is assumed to behave as a static linear system within the relevant operating range. Neglecting
the nonlinear behaviour close to the origin, the relationship between pump input voltage
$U_{\mathrm{in}}$ and mass flow $\dot{m}$ is approximated by
\[
    \dot{m} = K_{\mathrm{pump}} \left( U_{\mathrm{in}} - U_0 \right),
\]
where $K_{\mathrm{pump}}$ denotes the pump gain and $U_0$ represents the voltage offset. Both
parameters are obtained by linear regression of the measured characteristic curve. This model
is used both for simulation and for converting controller outputs into voltage signals during
implementation.

\paragraph{Pressure sensor model}
Each reservoir is equipped with a pressure sensor measuring the hydrostatic pressure at the
bottom of the tank. Since pressure is directly proportional to the water column height, the
sensor output voltage provides an indirect measurement of the water level. The sensor
characteristic is assumed to be linear and is described by
\[
    u_{\mathrm{out}} = k_h \, h + u_{\mathrm{off}},
\]
where $k_h$ is the sensor gain and $u_{\mathrm{off}}$ is an offset voltage. As only the water
level of the lower reservoir is controlled, only this sensor is considered in the control
loop.

\paragraph{Sensor calibration}
Accurate water level measurement requires calibration of the pressure sensor prior to control
design. Calibration is performed by recording the sensor output voltage at several known water
heights. Using these reference points, the gain and offset parameters are identified via
linear regression. The calibrated model enables conversion of the measured voltage signal into
a physical water height with sufficient accuracy for control purposes.

\paragraph{Identification of hydraulic parameters}
The hydraulic parameters governing the outflow of the reservoirs are summarized in the
coefficient
\[
    a = \frac{A_{\mathrm{out}}}{A_t}
    \sqrt{\frac{2g}{1+\zeta}},
\]
which combines outlet area, tank cross-sectional area, gravitational acceleration, and the
pressure loss coefficient. While geometric parameters and fluid density are known, the
effective pressure loss is identified experimentally. This is achieved by measuring the
steady-state outflow behaviour for different water levels and fitting the resulting data to
the theoretical flow model.

\paragraph{Practical constraints}
Several practical limitations influence system operation and controller design. The pump
input voltage is limited to a maximum admissible range to prevent hardware damage. The water
level in each reservoir is constrained between zero and the height of the emergency outlet,
which defines the maximum usable operating point. These constraints are explicitly considered
in simulations through saturation blocks and output limitations and must be respected during
laboratory operation.

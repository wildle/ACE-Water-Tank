\chapter{Simulation}
\label{chapter:Simulations}
\enlargethispage{3.5\baselineskip}

Figure~\ref{fig:bsbf} shows the complete Simulink model used for simulation of the closed-loop
water tank system. The nonlinear plant is implemented according to the state equations derived
in Chapter~\ref{chapter:TheoreticalBackground}. The states correspond to the water levels of the
upper and lower reservoirs and are obtained by numerical integration of the level dynamics.

\begin{figure}[H]
	\centering
	\includegraphics[width=\textwidth]{img/blockschaltb.png}
	\caption{Schematic of the Simulink simulation model}
	\label{fig:bsbf}
\end{figure}

The control structure combines a feedback controller with a feed-forward term. A reference
mass flow is generated by the feed-forward block based on the desired water level trajectory,
while disturbances and modeling inaccuracies are compensated by the feedback controller. The
resulting control signal is converted into a pump voltage using the identified actuator model
and applied to the nonlinear plant.

Non-ideal effects are included to reflect realistic operating conditions. The pump input
voltage is limited by a saturation block to respect actuator constraints. Measurement noise
is added to the water level signal using band-limited white noise. Furthermore, the water
levels are constrained to remain within the physically admissible range of the laboratory
setup.

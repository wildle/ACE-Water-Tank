\chapter{Theoretical Background}

\begin{figure}[!htbp]
    \centering
    \includegraphics[width=0.40\textwidth]{img/WaterTank_schematic.png}
    \caption{Schematic representation of the two-tank hydraulic system \cite{ACE_WT_Lab}}
    \label{fig:wt_physical}
\end{figure}

The water tank system is modeled as a nonlinear dynamic system consisting of two interconnected
reservoirs and a pump-driven inflow. Water enters the upper reservoir and flows into the lower
reservoir due to gravity. A detailed physical description of the system and the corresponding
parameter definitions are provided in the laboratory assignment \cite{ACE_WT_Lab}.

Assuming incompressible fluid behaviour, the nonlinear system dynamics can be expressed in
state-space form as
\begin{equation}
\dot{x}(t) = f(x(t),u(t)), \qquad
x = \begin{bmatrix} h_1 & h_2 \end{bmatrix}^\top,
\end{equation}
with the pump mass flow as control input and the water level of the lower reservoir as system
output.

For controller design, the nonlinear model is linearized around a steady operating point,
yielding a local linear time-invariant approximation. This model is used for PID controller
design. In addition, a flatness-based feed-forward approach is considered, exploiting the fact
that the lower reservoir level constitutes a flat output of the system.

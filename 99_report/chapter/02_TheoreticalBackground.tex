\chapter{Theoretical Background}
\label{chapter:TheoreticalBackground}
\enlargethispage{3.5\baselineskip}

\begin{figure}[H]
    \centering
    \includegraphics[width=0.40\textwidth]{img/WaterTank_schematic.png}
    \caption{Schematic representation of the two-tank hydraulic system \cite{ACE_WT_Lab}}
    \label{fig:wt_physical}
\end{figure}

The water tank system is modeled as a nonlinear dynamic system consisting of two interconnected
reservoirs and a pump-driven inflow. Water enters the upper reservoir and flows into the lower
reservoir due to gravity. A detailed physical description of the system and the corresponding
parameter definitions are provided in the laboratory assignment \cite{ACE_WT_Lab}.

Assuming incompressible fluid behaviour, the nonlinear system dynamics can be expressed in
state-space form as
\begin{equation}
\dot{\boldsymbol{x}}(t) = f(x(t),u(t)), \qquad
\boldsymbol{x} = \begin{bmatrix} 
		h_1 \\
		h_2 
	\end{bmatrix}
\end{equation}
where the control input $u(t)$ represents the pump mass flow. The controlled output is defined
as the water level of the lower reservoir,
\[
y(t) = h_2(t).
\]
\section{State linearization}
For controller design, the nonlinear model is linearized around a steady operating point,
resulting in a local linear time-invariant approximation. In addition to feedback control,
a flatness-based feed-forward approach is considered, exploiting the fact that the water level
of the lower reservoir constitutes a flat output of the nonlinear system.

\begin{align}
	\dot{x}_{1} &= -a_1\sqrt{x_1} + \frac{1}{\rho A_t}u \\
	\dot{x}_{2} &= a_1\sqrt{x_1} - a_2\sqrt{x_2}
\end{align}

where the parameters are defined as $a_{i}=\frac{A_{out,i}}{A_{t}}\sqrt{\frac{2g}{1+\zeta_{i}}}$ for $i=1,2$.


The nonlinear state-space model is linearized at an operating point determined by the desired water level in the second reservoir. As per the task description, this is chosen as $\bar{x}_2 = \qty{5}{\cm}$. The equilibrium conditions $\dot{\boldsymbol{x}} = 0$ yield:

\begin{equation}
	\bar{x}_{2} = \qty{0.05}{\meter}, \quad \bar{x}_{1} = \left(\frac{a_2}{a_1}\right)^2 \bar{x}_2, \quad \bar{u} = \rho A_t a_1 \sqrt{\bar{x}_1}
\end{equation}

The partial derivatives of the non-linear terms $f(x,u)$ evaluated at this operating point are:

\begin{align}
	\frac{\partial f_1}{\partial x_1} \bigg|_{\bar{x}} &= -\frac{a_1}{2\sqrt{\bar{x}_1}} \\
	\frac{\partial f_2}{\partial x_1} \bigg|_{\bar{x}} &= \frac{a_1}{2\sqrt{\bar{x}_1}} \\
	\frac{\partial f_2}{\partial x_2} \bigg|_{\bar{x}} &= -\frac{a_2}{2\sqrt{\bar{x}_2}}
\end{align}

Introducing deviations $\Delta \boldsymbol{x} = \boldsymbol{x} - \bar{\boldsymbol{x}}$ and $\Delta u = u - \bar{u}$, the linearized system $\Delta \dot{\boldsymbol{x}} = \boldsymbol{A}\Delta \boldsymbol{x} + \boldsymbol{B}\Delta u$ is given by:

\begin{equation}
	\begin{bmatrix}
		\Delta \dot{x}_1 \\
		\Delta \dot{x}_2
	\end{bmatrix}
	=
	\begin{bmatrix}
		-\frac{a_1}{2\sqrt{\bar{x}_1}} & 0 \\
		\frac{a_1}{2\sqrt{\bar{x}_1}} & -\frac{a_2}{2\sqrt{\bar{x}_2}}
	\end{bmatrix}
	\begin{bmatrix}
		\Delta x_1 \\
		\Delta x_2
	\end{bmatrix}
	+
	\begin{bmatrix}
		\frac{1}{\rho A_t} \\
		0
	\end{bmatrix}
	\Delta u
\end{equation}


\chapter{Conclusion and Outlook}
\label{chapter:Conclusion}
\enlargethispage{3.5\baselineskip}

This laboratory investigated the control of a nonlinear two-tank hydraulic system with the
objective of regulating the water level of the lower reservoir and enabling smooth level
transitions. A nonlinear model of the system was derived and linearized around a defined
operating point, providing the basis for controller design. Using identified system parameters,
a combined feedback and flatness-based feed-forward control structure was implemented and
evaluated in simulation and on the real laboratory setup.

Simulation results demonstrated accurate and smooth trajectory tracking under idealized
conditions, while experiments on the physical system confirmed the feasibility of the control
concept. Deviations between simulation and experiment were primarily caused by measurement
noise, actuator saturation and unmodeled hydraulic dynamics. Nevertheless, the overall system
response remained stable and smooth, indicating that the chosen modeling and control approach
is well suited for the considered application.

Future work may focus on improving tracking performance under non-ideal conditions. Possible
extensions include gain scheduling to account for varying operating points, the application of
nonlinear control strategies to better exploit the system dynamics and refined parameter
identification to reduce model mismatch. Additional improvements could be achieved through
enhanced sensor filtering and more accurate modeling of actuator dynamics.

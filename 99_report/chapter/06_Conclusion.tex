\chapter{Conclusion}
\label{chapter:Conclusion}
\enlargethispage{3.5\baselineskip}

This laboratory investigated the control of a nonlinear two-tank hydraulic system with the goal
of regulating the water level of the lower reservoir while enabling smooth level transitions.
A nonlinear model of the hydraulic dynamics was derived and linearized around a defined
operating point, forming the basis for controller design. System parameters were identified
experimentally and used consistently for simulation and real-time implementation.

Simulation results demonstrated smooth and accurate trajectory tracking under idealized model
conditions. When applied to the physical laboratory setup, the control strategy remained stable
and achieved smooth level transitions, confirming the practical feasibility of the combined
feedback and flatness-based feed-forward approach. Differences between simulation and
experiment were observed in the form of small tracking deviations and steady-state offsets.

These deviations were primarily caused by measurement noise, actuator saturation and
unmodeled hydraulic effects that are not fully captured by the simplified model. Despite these
limitations, the control concept proved robust against practical non-idealities and fulfilled
the control objectives defined for the laboratory experiment.

\chapter{Results and Interpretation}
\label{chapter:ResultsAndInterpretation}
\enlargethispage{3.5\baselineskip}

Figure~\ref{fig:simtrans} shows the simulated height response of the water tank system using the
nonlinear model and the parameters identified in Chapters~2 and~3. A smooth polynomial
reference trajectory is applied to generate transitions between two operating levels of the
lower reservoir. Both upward and downward transitions are obtained by adjusting the boundary
conditions of the trajectory.

The simulated water level closely follows the reference trajectory throughout the transition.
A smooth response without overshoot is observed, while only small deviations occur due to the
injected measurement noise. Under these idealized conditions, the combined feedback controller
and flatness-based feed-forward achieve accurate trajectory tracking. The simulation results
therefore provide a suitable reference for evaluating the control strategy on the real
laboratory system.


\begin{figure}[H]
	\centering
	\includegraphics[width=\textwidth]{img/wt_sim_traj.pdf}
	\caption{Simulated water level response of the lower reservoir for a reference trajectory transition}
	\label{fig:simtrans}
\end{figure}

Figure~\ref{fig:realtrans} presents the measured height response of the water tank system for the
same reference trajectory applied in simulation. In contrast to the idealized simulation result,
the real system exhibits small tracking deviations during the transition phases and minor
steady-state offsets.

Despite these deviations, the overall response remains smooth, demonstrating that the
trajectory planning and feed-forward approach are effective also on the real hardware.
Measurement noise, actuator saturation and unmodeled hydraulic dynamics contribute to the
reduced tracking accuracy compared to the simulation. In particular, pump limitations and
model mismatches influence the transient behavior during rapid level changes.

Overall, the experimental results confirm the validity of the control concept developed in
simulation, while highlighting the impact of practical non-idealities inherent to the physical
system.

\begin{figure}[H]
	\centering
	\includegraphics[width=\textwidth]{img/wt_realsystem_traj.pdf}
	\caption{Measured water level response of the lower reservoir for the same reference trajectory}
	\label{fig:realtrans}
\end{figure}

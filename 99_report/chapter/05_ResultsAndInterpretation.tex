\chapter{Results and Interpretation}
\enlargethispage{3.5\baselineskip}

!!! Text ändern !!!

Figure~\ref{fig:simtrans} shows the simulated height response of the Ball-in-Tube system using the
nonlinear model and the parameters identified in Chapters~2 and~3. A seventh-order polynomial
trajectory is used to generate smooth transitions between the lower height of
\qty{0.1}{\meter} and the upper height of \qty{0.4}{\meter}. The downward motion is obtained by
reversing the boundary conditions of the trajectory.

The simulated height closely follows the reference trajectory during both the upward and downward
transitions. The response is smooth and does not exhibit overshoot, while only minor deviations
occur due to the injected measurement noise. Overall, the results indicate that the PID controller
and the flatness-based feed-forward provide satisfactory tracking performance in simulation.
These results serve as a reference for evaluating the behavior of the control strategy on the real
system.
\begin{figure}[H]
	\centering
	\includegraphics[width=\textwidth]{img/plotsimsystemtraj.pdf}
	\caption[Transition in the Simulation]{Transition from \qty{0.1}{\meter} to \qty{0.4}{\meter} using the Simulink-Simulation and the non-linear model of the system}
	\label{fig:simtrans}
\end{figure}
Figure~\ref{fig:realtrans} shows the measured height response of the Ball-in-Tube system for the
same reference trajectory used in the simulation. In contrast to the idealized simulation result,
the real system exhibits small tracking deviations during the transition phases and a slight
steady-state offset during the plateau.

The upward and downward motions remain smooth, indicating that the trajectory planning and
feed-forward approach are effective also on the real hardware. However, measurement noise,
actuator limitations, and unmodeled dynamics lead to a reduced tracking accuracy compared to
the simulation. In particular, the finite fan dynamics and input saturation influence the transient
response.

Overall, the real-system results confirm the validity of the control concept derived in simulation,
while highlighting the impact of practical non-idealities that are not fully captured by the model.
\begin{figure}[H]
	\centering
	\includegraphics[width=\textwidth]{img/plotrealsystemtraj.pdf}
	\caption[3D-Modell einer Offline-Simulation]{3D-Modell einer Offline-Simulation, Bildquelle: Visual Components GmbH}
	\label{fig:realtrans}
\end{figure}

